% !TEX TS-program = xelatex
% !TEX encoding = UTF-8 Unicode
% !Mode:: "TeX:UTF-8"

\documentclass{resume}
\usepackage{zh_CN-Adobefonts_external} % Simplified Chinese Support using external fonts (./fonts/zh_CN-Adobe/)
%\usepackage{zh_CN-Adobefonts_internal} % Simplified Chinese Support using system fonts
\usepackage{linespacing_fix} % disable extra space before next section
\usepackage{cite}

\begin{document}
\pagenumbering{gobble} % suppress displaying page number
\hypersetup{hidelinks}
\name{于行尧}

\basicInfo{
  \email{yuxy16@mails.tsinghua.edu.cn} \textperiodcentered\ 
  \phone{(+86) 137-2557-4186} \textperiodcentered\ 
  \homepage[yaoskingdom]{www.yaoskingdom.tk}}
 
\section{\faGraduationCap\  教育背景}
\datedsubsection{\textbf{清华大学}, 北京}{2016 -- 至今}
\textit{保送硕士研究生}\ 控制科学与工程, 主修课程包括矩阵分析、凸优化、随机过程等
\datedsubsection{\textbf{北京航空航天大学}, 北京}{2012 -- 2016}
\textit{学士}\ 飞行器设计与工程(航天),专业课程包括卡尔曼滤波、数电模电、数字信号处理等

\section{\faUsers\ 实习经历}
\datedsubsection{\textbf{工业机器人无序分拣}}{2018年5月 -- 至今}
\role{助理研究员}{商汤科技研究院}
\begin{onehalfspacing}
使用ENSENSO 3D相机获取点云数据,控制欧姆龙机器人实现目标零件(欧姆龙零件、鼠标滚轮)的定位与抓取
\begin{itemize}
  \item 阅读相机文档,实现传感器数据获取
  \item 使用开源框架(PCL)进行图像预处理、特征提取、场景与模型的匹配、位姿估计
  \item 使用欧姆龙viper机器人抓取零件(ROS)
  \item 完成其他工作,如设计制作了Intel RealSense 415相机的安装支架等机械结构零件(CATIA)
\end{itemize}
\end{onehalfspacing}

\section{\faFlag\ 项目经历}
\datedsubsection{\textbf{超冗余连续型柔性机械臂的建模及运动规划研究}}{2017年9月 -- 至今}
\role{研究生工作}{清华大学}
\begin{onehalfspacing}
该项目承接中央军委科技委重大项目,对于超冗余柔性机械臂的建模及运动规划展开研究。
空间环境多为复杂非结构环境,并且设备价格高昂,因而对于进行在轨操作的机器人的狭小空间作业能力及柔顺性
提出了很高的要求。本项目中研究的超冗余连续性机械臂由于自由度极高,运动灵活,并且具有良好的力柔顺性,
特别适合这类空间在轨操作。
\begin{itemize}
  \item 独立制作了一套柔性臂实验系统,采用PyQt编写了图形控制界面,C语言编程使用ARM同时控制15个电机运动
  \item 利用模态函数、伪逆法、RRT等方法对运动规划展开研究
  \item 已发表多篇文章和专利
\end{itemize}
\end{onehalfspacing}

\datedsubsection{\textbf{基于一致性理论的四旋翼编队飞行研究}}{2015年12月 -- 2016年6月}
\role{本科毕业设计}{北京航空航天大学}
\begin{onehalfspacing}
四旋翼飞行器的真机编队飞行
\begin{itemize}
  \item 实验环境为MotionCap,上位机为Raspberry Pi。通信采用UDP
  \item 实现了三架小型无人机的编队飞行
  \item Matlab仿真环境下针对非合作目标的编队跟随进行了方法研究
\end{itemize}
\end{onehalfspacing}

% Reference Test
%\datedsubsection{\textbf{Paper Title\cite{zaharia2012resilient}}}{May. 2015}
%An xxx optimized for xxx\cite{verma2015large}
%\begin{itemize}
%  \item main contribution
%\end{itemize}

\section{\faCogs\ IT/工程技能}
% increase linespacing [parsep=0.5ex]
\begin{itemize}[parsep=0.5ex]
  \item 编程语言: Python、 C、 C++
  \item 科研软件: Matlab、Adams、Catia、V-Rep、ROS
  \item 开发: Linux、tensorflow、ARM、OpenCV、Latex
\end{itemize}

\section{\faHeartO\ 获奖情况}
%\datedline{\textit{第一名}, xxx 比赛}{2013 年6 月}
\datedline{综合优秀奖,\textit{清华大学}}{2017}
\datedline{国家奖学金,\textit{北京航空航天大学}}{2015}
\datedline{国家励志奖学金,\textit{北京航空航天大学}}{2014}
\section{\faInfo\ 其他}
% increase linespacing [parsep=0.5ex]
\begin{itemize}[parsep=0.5ex]
  \item 已录用论文:Xingyao Y, Xueqian W, Bin L, et al. Differential Kinematics for A Tendon-driven Snake-like Robot[C]//Control and Decision Conference (CCDC), 2018 30th Chinese. IEEE
  \item 审查中论文:Xingyao Y,Xueqian W, Bin L, et al. Collision Free Path Planning for Multi-Section Continuum Manipulators Based on A Modal Method. [C]//IEEE-CYBER 2018
  \item 申请中专利:18A100057JYC-一种球铰绳驱柔性机械臂专利技术\\
  \hspace*{6em}7A100544JYC-一种连续型机械臂的空间避障轨迹规划方法\\
  \hspace*{6em}一种新型绳驱蛇形机器人的建模方法及运动学求解方法
  \item 语言: 英语 - 熟练(六级 625)
  			
\end{itemize}

%% Reference
%\newpage
%\bibliographystyle{IEEETran}
%\bibliography{mycite}
\end{document}
